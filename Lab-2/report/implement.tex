\section{实验实现方法}
在本次实验中,由于DFA和LR Table最终都可以用一张表来表示,所以采用csv文件来存储用户输入的LR Table和DFA,以实现软编码的目的。然后根据读入的表进行解析,首先用Lexer读入用户输入的源文件,一次读入一个字符,并在DFA上进行状态之间的转换,并适当的进行报错处理。Lexer提供了next函数,每次返回一个Token类,用来表示识别到的词,然后交由Parser处理,Parser根据传入的Token和LR Table进行状态的移动,最终形成一棵Parser Tree
\subsection{SoftCodeLexer关键内容描述}
\subsubsection{State}
该类主要内容为String name,boolean isAcceptState以及HashMap<String, String> mapping2statename;它们分别记录了当前状态的名称,是否为结束节点,以及该状态能经过什么符号到达下一个状态的名称;


并且提供了方法getNextStateName,它能根据传入的字符寻找能到达的状态的名称并返回,如果没有则返回null
\subsubsection{DFA}
该类主要保存用户输入的DFA,其中关键的数据成员为:HashMap<String, State> name2state,该map用于记录所有的State并将它们的名称和状态做映射;State start用于记录开始状态;State currentState用于DFA在状态之间转换时记录当前所在的状态;


该类提供了next函数,它接受一个字符,并判断能否从当前状态经过这个字符到达另一个状态,如果不可以则返回false,可以的话返回true;同时提供了函数isAcceptingState,用来判断当前状态是否是accept的,还提供了returnToStart函数,用于将DFA的当前状态重置到开始状态。


对DFA做这样的封装,可以让外部使用者只调用next,判断是否是accepting的即可,对于DFA内部的状态是如何转换的可以不关心。

\subsubsection{Token}
该类用于表示词法分析器生成的Token,主要含有Long id, String type, String lexValue,分别表示该Token的id,类型和对应的解析出来的真实值。其中type为用户在配置文件中自己定义的名称。
\subsubsection{Lexer}
该类为实现词法分析的类,它使用封装好的DFA,并维护了一个char[] buf,buffer的大小可由用户配置,其工作流程为每次读入buffer大小的字符,对字符进行遍历,遍历一个字符将其交给DFA,如果DFA返回false,说明不能继续走,然后判断DFA当前状态能否接受,如果能则输出,不能则报错。

\subsection{SoftCodeParser关键内容描述}
\subsubsection{Expression}
该类主要用于存储用户输入的语法产生式,由于我们要求的产生式为上下文无关文法,所以该类用一个String leftPart存储产生式的左部,而右部用ArrayList<String> rightPart来按照顺序存储
\subsubsection{Grammar}
该类用于存储所有的产生式,它使用一个HashMap<String, Expression> allExpression将用户定义的产生式名称和产生式相关联并存储,并提供了按照名称返回产生式的方法。
\subsubsection{LRTableRow}
该类用于存储LR Table中的一行,类似于DFA中的状态,它主要含有两个数据成员:HashMap<String, Pair<String, Integer>> actions,用来存储action表,它将表示该状态能从哪种Token转移到下一个状态或用什么来规约,Pair<String, Integer>种用Integer表示是移进还是规约还是到达了accept状态,如果是移进,则Pair中第一个String表示要移进的状态名,如果是规约或者accept则表示产生式的名称;HashMap<String, String> goto表示goto表,表示能从哪个非终结状态到达哪个状态;String name代表当前行的状态名称。
\subsubsection{LRTable}
该类用来存储完整的LRTable,利用HashMap<String, LRTableRow> rows存储所有行,并将每行的状态的名称和该行做映射,并提供了canReach和canGoto函数用来判断通过给定的终结符或非终结符的下一步状态,如规约,移进或accept。
\subsubsection{ParserTreeNode}
该类是生成的语法分析树的节点,保存了int id,boolean isLeaf表示是否为叶节点,String symbol表示对应Token的type,String value表示如果是叶节点则它的lexValue是多少,以及List<ParserTreeNode>表示它的儿子节点
\subsubsection{ParserTree}
该类保存了ParserTree的根节点
\subsubsection{Parser}
该类为语法分析器的核心,它使用了Lexer和LRTable,parser函数为解析函数,它每次读入Lexer的一个字符,同时维护符号栈和状态栈,根据Token的type查LRTable决定下一步动作:如果是移入则创建新的ParserTreeNode,将Token中的lexValue放入,并令其为叶节点,压入符号栈;如果规约则相应的将符号栈中的符号弹出与Grammar中的Expression比较能否成功规约,并创建新的ParserTreeNode,其不是叶节点,并且儿子节点为符号栈中弹出的可规约的ParserTreeNode,规约完成后将这个节点压入符号栈,并根据goto表查找正确的状态名称压入状态栈。


这样在规约结束时即可获得一棵语法分析树。