\section{实验小结}
通过本次实验,对编译原理中的词法分析部分有了较深的理解。在实验过程中,一开始遇到的阻力很大,比如最开始的.l文件的读取处理,由于不想在这个部分花费太多时间,想把精力集中在后面的算法部分,就利用xml文件来组织用户输入的正规表达式,并利用第三方库来解析。同时对于正规表达式规范化部分也遇到了很多问题,甚至去github上查看了flex的源代码,最后利用自己规定的正规表达式规则,并且用字符串的搜索匹配方法,对正规表达式不断扫描,每次扫描进行一种处理,最终将正规表达式转换为规范化的正规表达式。


对于中缀转后缀,后缀转NFA和NFA转DFA以及DFA最小化,有确定的算法,写起来比较轻松。

此外还遇到的一个问题就是对于字符的处理,因为我希望这个程序能处理多种字符,而不只是处理ASCII中的字符,所以提供了一个charset的接口,可以让用户自己配置自己词法分析器中的字符集,比如可以配置UNICODE字符集等,但由于时间精力的原因,目前只提供了ASCII的字符集实现,但由于使用了接口,使得程序易于扩展,只要添加相应的实现类以及配置选项即可。